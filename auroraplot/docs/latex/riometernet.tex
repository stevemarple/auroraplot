\documentclass{article}
\usepackage[utf8]{inputenc}
 
\usepackage{listings}
\usepackage{color}
\usepackage[T1]{fontenc}
\usepackage{lmodern}
 
\definecolor{codegreen}{rgb}{0,0.6,0}
\definecolor{codegray}{rgb}{0.5,0.5,0.5}
\definecolor{codepurple}{rgb}{0.58,0,0.82}
\definecolor{backcolour}{rgb}{0.95,0.92,0.92}
 
\lstdefinestyle{mystyle}{
    backgroundcolor=\color{backcolour},   
    %commentstyle=\color{codegreen},
    %keywordstyle=\color{magenta},
    %numberstyle=\tiny\color{codegray},
    %stringstyle=\color{codepurple},
    basicstyle=\footnotesize\ttfamily,
    basewidth={0.5em,0.5em},
    breakatwhitespace=false,         
    breaklines=true,                 
    captionpos=b,                    
    keepspaces=true,                 
    numbers=left,                    
    numbersep=5pt,                  
    showspaces=false,                
    showstringspaces=false,
    showtabs=false,                  
    tabsize=4
}
 
\lstset{style=mystyle}
 
\begin{document}

\title{Getting started with Auroraplot}
\date{}
\maketitle

\section{Minimum requirements}
Auroraplot has been tested on Fedora 24, Ubuntu 16.04 LTS, and Debian 8.
It should work on Apple OSX with minimal changes to the configuration steps.
Microsoft Windows is not currently supported.

Loading and manipulating large data sets requires significant amounts of RAM.
On systems with 8 GB frequent swapping to disk and slow-downs can occur,
especially when creating Quiet Day Curves. 16 GB or more is recommended.

\section{Installing Auroraplot}

Install the dependencies. In a terminal type

\begin{figure}[htb!]
\begin{minipage}[b]{0.45\linewidth}
Fedora 24
\begin{lstlisting}[language=Bash]
sudo dnf install python-pyside git python2-matplotlib-qt4 python-requests ipython python-scipy python-pandas
\end{lstlisting}
\end{minipage}
\hspace{0.5cm}
\begin{minipage}[b]{0.45\linewidth}
Debian 8 / Ubuntu 16.04
\begin{lstlisting}                 
sudo apt-get install git python-matplotlib python-scipy 
\end{lstlisting}
\end{minipage} 
\end{figure}

Pandas is an optional dependency (54.8 MB) that will speed up the loading of data.

\subsection{Installing the latest version of Auroraplot}
To install Auroraplot in your home folder type
\begin{lstlisting}[language=Bash]
cd ~
git clone https://github.com/m-j-b/auroraplot.git
\end{lstlisting}

Python looks in the in the site-packages folder for installed python packages. Create the folder, and in it make a symlink to the Auroraplot package.
\begin{lstlisting}[language=Bash]
mkdir -p ~/.local/lib/python2.7/site-packages
ln -s ~/auroraplot/auroraplot ~/.local/lib/python2.7/site-packages/
\end{lstlisting}

To update to the latest version of Auroraplot run
\begin{lstlisting}[language=Bash]
cd ~/auroraplot/
git fetch --all
git checkout --force master
\end{lstlisting}


\section{Running Auroraplot in IPython}

By default, scripts that are imported into python and subsequently edited will not be automatically reloaded.
To change this behaviour, create a startup file for IPython.

\begin{lstlisting}[language=Bash]
mkdir -p ~/.ipython/profile_default/startup/
nano ~/.ipython/profile_default/startup/10-custom.ipy
\end{lstlisting}

In the file enter (including the \% symbols)
\begin{lstlisting}[language=python]
%load_ext autoreload
%autoreload 2
%pylab
\end{lstlisting}

\subsection{IPython Examples}

To speed up loading of data, copy the example riometer data (1.5 GB) to the /data directory, which is the default location.

\begin{lstlisting}[language=Bash]
sudo mkdir /data
sudo wget -m -np -nH --cut-dirs=1 -P=/data http://www.riometer.net/example_data/2004
\end{lstlisting}

Create a directory for Quiet Day Curves (QDCs), and a group (qdc) for users that have permission to create and modify QDCs in that directory.

\begin{lstlisting}[language=Bash]
sudo groupadd qdc
sudo mkdir /data/qdc
sudo chown root:qdc /data/qdc
sudo chmod g+w /data/qdc
\end{lstlisting}

Add the current user to the qdc group, if required.

\begin{lstlisting}[language=Bash]
sudo usermod -aG qdc $USER
\end{lstlisting}

It is helpful to see the logger output in IPython. This will indicate errors and failures to load data. To have log messages display on the screen run

\begin{lstlisting}[language=Python]
import logging
logging.basicConfig(stream=sys.stdout, level=logging.DEBUG)
\end{lstlisting}


\subsubsection{Example 1: Plotting riometer power data}

In this example, we load one day of riometer power data from beam 25 of the Kilpisjarvi riometer, KIL1 (IRIS). The example data covers Jan 2004, but we will load and plot data from the 15th Jan 2004.

\begin{lstlisting}[language=Python]
# import the riodata module, and the datasets metadata
import auroraplot as ap
import auroraplot.riodata as riodata
import auroraplot.datasets.riometernet

# set the start and end times
st = np.datetime64('2004-01-15T00:00:00')
et = st + np.timedelta64(1,'D')

# load the power data
rd = ap.load_data('RN','KIL1','RioPower',st,et,'local archive',['25'])

# plot the data
rd.plot()
\end{lstlisting}

Hints on how to use functions such as {\it ap.load\_data} are seen by appending the function with a question mark.

\begin{lstlisting}[language=Python]
ap.load_data?

Definition:  ap.load_data(project, site, data_type, start_time, end_time, archive=None, channels=None, path=None, load_function=None, raise_all=False, cadence=None, aggregate=None, filter_function=None)
\end{lstlisting}

Each riometer has a site name (in this case 'KIL1'), and belongs to a particular project. 'RN' stands for the Riometer Network project. The archive name 'local archive' could be changed to 'remote archive' in order to load this example data from www.riometer.net. Sites, projects and archives are defined in the datasets module ({\it auroraplot.datasets.riometernet}).

Note that the channels are specified as a list of strings. When dealing with riometer data, channels (beams) are usually numbered from 1, but magnetometer channels may be named 'X', 'Y', and 'Z', or 'H', 'D', and 'Z'.


\subsubsection{Example 2: Making Quiet Day Curves}

To make a quiet day curve, it is necessary to load more than one day of data. Usually 14 days will suffice. By default, the valid period for a QDC will be 14 days long.
Auroraplot has functions in the ap.dt64tools package to find the standard boundaries for the QDCs: {\it ap.dt64tools.floor, and ap.dt64tools.ceil}.
Assuming we have already imported auroraplot as ap.

\begin{lstlisting}[language=Python]
t = np.datetime64('2004-01-15T00:00:00')
st = ap.dt64tools.floor(t, np.timedelta64(14,'D'))
et = ap.dt64tools.ceil(t, np.timedelta64(14,'D'))
rd = ap.load_data('RN','KIL1','RioPower',st,et,'local archive',channels=['25'])

\end{lstlisting}

{\it st} holds the datetime64 value of '2004-01-12T00:00:00', and {\it et} is '2004-01-26T00:00:00'.

The QDC is made from the power data by calling.

\begin{lstlisting}[language=Python]
qdc = rd.make_qdc()
qdc.plot(channels=['25'])
\end{lstlisting}

Calling make\_qdc() will create QDCs for all channels of the riometer (according to the metadata in auroraplot.datasets). But since {\it rd} only contains data for channel '25', calling {\it qdc.plot()}, without the {\it channels} argument will make mostly empty plots.

If the computer has sufficient RAM ($\ge 12$ GB recommended) data from all channels can be loaded at once.


\begin{lstlisting}[language=Python]
rd = ap.load_data('RN','KIL1','RioPower',st,et,'local archive')
qdc = rd.make_qdc()
qdc.plot()
\end{lstlisting}


\end{document}

