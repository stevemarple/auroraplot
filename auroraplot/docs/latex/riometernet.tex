\documentclass{article}
\usepackage[utf8]{inputenc}
 
\usepackage{listings}
\usepackage{color}
\usepackage[T1]{fontenc}
\usepackage{lmodern}
 
\definecolor{codegreen}{rgb}{0,0.6,0}
\definecolor{codegray}{rgb}{0.5,0.5,0.5}
\definecolor{codepurple}{rgb}{0.58,0,0.82}
\definecolor{backcolour}{rgb}{0.95,0.92,0.92}
 
\lstdefinestyle{mystyle}{
    backgroundcolor=\color{backcolour},   
    %commentstyle=\color{codegreen},
    %keywordstyle=\color{magenta},
    %numberstyle=\tiny\color{codegray},
    %stringstyle=\color{codepurple},
    basicstyle=\footnotesize\ttfamily,
    basewidth={0.5em,0.5em},
    breakatwhitespace=false,         
    breaklines=true,                 
    captionpos=b,                    
    keepspaces=true,                 
    numbers=left,                    
    numbersep=5pt,                  
    showspaces=false,                
    showstringspaces=false,
    showtabs=false,                  
    tabsize=4
}
 
\lstset{style=mystyle}
 
\begin{document}

\title{Getting started with auroraplot}
\date{}
\maketitle

\section{Minimum requirements}
Auroraplot has been tested on Fedora 24, Ubuntu 16.04 LTS, and Debian 8.
It should work on Apple OSX with minimal changes to the configuration steps.
Microsoft Windows is not currently supported.

Loading and manipulating large data sets requires significant amounts of RAM.
On systems with 8 GB frequent swapping to disk and slow-downs can occur,
especially when creating Quiet Day Curves. 16 GB or more is recommended.

\section{Installing Auroraplot}

Install the dependencies. In a terminal type

\begin{figure}[htb!]
\begin{minipage}[b]{0.45\linewidth}
Fedora 24
\begin{lstlisting}[language=Bash]
sudo dnf install python-pyside git python2-matplotlib-qt4 python-requests ipython python-scipy python-pandas
\end{lstlisting}
\end{minipage}
\hspace{0.5cm}
\begin{minipage}[b]{0.45\linewidth}
Debian 8 / Ubuntu 16.04
\begin{lstlisting}                 
sudo apt-get install git python-matplotlib python-scipy 
\end{lstlisting}
\end{minipage} 
\end{figure}

Pandas is an optional dependency (54.8 MB) that will speed up the loading of data.

\subsection{Installing the latest version of Auroraplot}
To install Auroraplot in your home folder type
\begin{lstlisting}[language=Bash]
cd ~
git clone https://github.com/m-j-b/auroraplot.git
\end{lstlisting}

Python looks in the in the site-packages folder for installed python packages. Create the folder, and in it make a symlink to the Auroraplot package.
\begin{lstlisting}[language=Bash]
mkdir -p ~/.local/lib/python2.7/site-packages
ln -s ~/auroraplot/auroraplot ~/.local/lib/python2.7/site-packages/
\end{lstlisting}

To update to the latest version of Auroraplot run
\begin{lstlisting}[language=Bash]
cd ~/auroraplot/
git fetch --all
git checkout --force master
\end{lstlisting}


\section{Running Auroraplot in IPython}

By default, scripts that are imported into python and subsequently edited will not be automatically reloaded.
To change this behaviour, create a startup file for IPython.

\begin{lstlisting}[language=Bash]
mkdir -p ~/.ipython/profile_default/startup/
nano ~/.ipython/profile_default/startup/10-custom.ipy
\end{lstlisting}

In the file enter (including the \% symbols)
\begin{lstlisting}[language=python]
%load_ext autoreload
%autoreload 2
%pylab
\end{lstlisting}

\subsection{IPython Examples}

To speed up loading of data, copy the example riometer data (1.5 GB) to the /data directory, which is the default location. Then start IPython

\begin{lstlisting}[language=Bash]
sudo mkdir /data
sudo wget -m -np -nH --cut-dirs=1 -P=/data http://www.riometer.net/example_data/2004
ipython
\end{lstlisting}

It is helpful to see the logger output in IPython. This will indicate errors and failures to load data. To show log messages on the screen run
\begin{lstlisting}[language=Python]
import logging
logging.basicConfig(stream=sys.stdout, level=logging.DEBUG)
\end{lstlisting}


\subsubsection{Example 1: Plotting riometer power data}

In this example, we load Kilpisjarvi power data for 1 Jan 2004, beam 25.

\begin{lstlisting}[language=Python]
# import the riodata module, and the datasets metadata
import auroraplot as ap
import auroraplot.riodata as riodata
import auroraplot.datasets.riometernet

# set the start and end times
st = np.datetime64('2004-01-01T00:00:00')
et = st + np.timedelta64(1,'D')

# load the power data
rd = ap.load_data('RN','KIL1','RioPower',st,et,'local archive',['25'])

# plot the data
rd.plot()
\end{lstlisting}


\end{document}

